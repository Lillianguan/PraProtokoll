
% Diese ist ein Template zum Erstellen der Protokoll zum Praktikum Ein standardisierter und optimierter Prozess zur Erschließung von digitalen Herbarbelegen
% Fachgebiet Elektrotechnik und Informationstechnik, Leibniz Universität Hannover 
% 30/05/2017
% Cailin Guan 

\documentclass[10pt,a4paper]{report}

%%%%%%%%%%%%%%%%%%%%%%%%%%%%%%%%%%%%%%%%%%%%%%%%%%%%%%
%% grnutzte Package
\usepackage{ragged2e}
\usepackage{epsfig}  
\usepackage[utf8]{inputenc}
\usepackage[ngerman]{babel}
\usepackage{titlesec}
\usepackage{graphicx}
\usepackage[style=numeric-comp,backend=biber]{biblatex}
\usepackage{csquotes}
%%%%%%%%%%%%%%%%%%%%%%%%%%%%%%%%%%%%%%%%%%%%%%%%%%%%%%
\addbibresource{MyBib.bib}

%% ein paar kleine Modifikationen am Format
\parindent0pt                        % erste Zeile eines Absatzes nicht einrcken
\parskip 1.5ex                       % Absatzabstand
\raggedbottom                        
\renewcommand{\baselinestretch}{1.3} % Zeilenabstand
%%%%%%%%%%%%%%%%%%%%%%%%%%%%%%%%%%%%%%%%%%%%%%%%%%%%%%
%%
\titleformat{\chapter}
{\normalfont\bfseries} % format
{}{0pt}{\large}           % before-code
\titleformat{\section}
{\normalfont\bfseries} % format
{\thesection}{1em}{}        
\titleformat{\subsection}
{\normalfont\bfseries}{\thesubsection}{1em}{}
\titlespacing{\chapter}{0pt}{-50pt}{0pt}  

%%%%%%%%%%%%%%%%%%%%%%%%%%%%%%%%%%%%%%%%%%%%%%%%%%%%%%

%% Title Page
\title{Praktikumsprotokoll}
\author{Cailin Guan}
\date{\today}

%%%%%%%%%%%%%%%%%%%%%%%%%%%%%%%%%%%%%%%%%%%%%%%%%%%%%%%
%% Hier beginnt dieses Doku
\begin{document}
\maketitle  
% Erklärung für dieses Praktikum
\chapter*{Erklärung}

Hiermit erkläre ich, dass ich die vorliegende Arbeit selbst andig und ohne fremde Hilfe verfasst, und dass nur die angegebenen Quellen und Hilfsmittel verwendet wurden. Die Arbeit wurde bischer in gleicher oder änhlicher Form keineranderen Prüfungsbehörde vorgelegt.\\

Hannover, 30. Mai 2017\\

\underline{\hspace{4.5cm}}\\
Unterschrift
% Inhaltsverzeichnis	
\tableofcontents  
 
\chapter{Einleitung}

Vom September 2014 bis Mai 2017 arbeite ich in Teilzeit(33\%) an der Hochschule Hannover im Forschungsprojekt \glqq Standard Daten Akquisitionsprozess \grqq, Abkürzung \glqq stanDAP-Herb \grqq mit Visual Studio und C/C++/C\#. Dazu gehört die Entwicklung automatisierter Erkennung von Handschriftlichen Herbar - Belegen. Unter der Berücksichtigung bestehender Datenbanksystem und die Entwicklung einer funktionalen webbasierten, anwendungsorientierten Bedienoberfläche. Meinen Aufgaben zählen die digitalen Bildverarbeitung mit OpenCV Bibliothek und Programmierung, z.B GUI Oberfläche zur Web-Lokal-Steuerung, Objekterkennung mit Features, Klassifikation von multiple Objekten, Maschine Learning und Datenanalysieren wie XML und Json.\\
Ich habe am \glqq 3+1\grqq Austausch- und Kooperationsprogramm der Hochschule Hannover und der Partner Zhejiang University of Science and Technology in Hangzhou teilgenommen, und kam im Rahmen dieses Programms im Herbst 2013 nach Deutschland und belegte im WS13/14 einige englischsprachige Veranstaltung. Meine Praktikum und Bachelorarbeit absolvierte ich in diesem Forschungsprojekt \glqq stanDAP\-Herb\grqq. Danach habe ich einen Chance von Prof. Dr. Karl-Heinz Steinke bekommen, in diesem Projekt weiter zu arbeiten.\\
Das Thema \glqq Bildverarbeitung\grqq interessiert mich seit erstem Besuchen der Vorlesung \glqq Digitale Bildverarbeitung\grqq. Deshalb freute ich mich sehr, als ich die Arbeitsstelle bekam, und ich so die Möglichkeit bekam, im Rahmen dieser Teilzeitbeschäftigung arbeite ich mit zwei andere Studenten im Team zusammen. Dabei habe ich Berufserfahrung gesammelt und selbstständiges, eigenverantwortliches Arbeiten gelernt.\\

\chapter{Rahmenbedingungen}
\section{Art,Inhalt und Umfang dem Thema}

Das Projekt gehört zu Forschungsprojekt, die Grund, warum es gemacht wird, lautet, bisher werden die Metadaten von Herbarbelegen manuell in Sammlungsdatenbanken eingegeben, aber zunehmend werden Bilderfassungsverfahren eingesetzt, die auch die Nachprüfbarkeit der online verfügbaren Metainformation sichern. Das Standardverfahren soll nun so weit wie möglich die manuelle Metadatenerfassung ersetzen oder ergänzen. Bildverarbeitungssoftware erkennt Objekte auf dem digitalisierten Herbarbeleg und klassifiziert sie. Die Textobjekte werden mit Hilfe von Text - Mining Algorithmen in strukturierte Information überführt. Bei Handschriften wird versucht, den Autor zu erkennen. Im Projekt wird vorhandene Software evaluiert, unter Bildung von standardisierten Interfaces weiterentwickelt und in eine übergreifende offene Softwarearchitektur auf Grundlage etablierter IT-Standards integriert. Abschließend wird das Verfahren hinsichtlich seiner Anforderungen als Standard formuliert und hinsichtlich seiner Anwendung dokumentiert. Das Verfahren adressiert einen großen Bereich naturwissenschaftlicher Sammlungen, allein in Deutschland liegen ca. 22 Millionen Herbarbelege vor, weltweit über 500 Mio \cite{1}.\\
\begin{figure}[htbp] 
	\centering
	\includegraphics[width=0.7\textwidth]{Herbarbeleg_Objekte.JPG}
	\caption{Beispiel zu digitalem Bild \cite{2}, auf Herbarbelegen werden Metadaten wie Artname, Fundort und - datum, Sammler, Katalognummern etc. mit Etiketten, Barcodes usw. flächig sichtbar auf den Bogen gebracht und damit im Foto oder Scan abgebildet.}
	\label{fig:Bild1}
\end{figure}


\section{Vorstellung des Projekt: StanDAP-Herb}

Das Projekt, Standard Daten Akquisitionsprozess - Ein standardisierter und optimierter Prozess zur Erschließung von digitalen Herbarbelegen, ist durch DFG Deutsche Forschungsgemeinschaft, Literaturversorgung und Information / Erschließung und Digitalisierung gefördert, dauert insgesamt 3 Jahre lang, ab Juli 2014 bis Juni 2017. Das Ziel von diesem Projekt ist, einen softwarebasierten Standardprozess für die Extraktion von Metadaten von digitalen Herbarbelegen zu entwickeln und dokumentieren. Die Partnern von diesem Projekt sind \textit{Botanic Garden and Botanical Museum Berlin}, \textit{Fraunhofer IOSB - Karlsruhe} uns \textit{Hochschule Hannover}.

\chapter{Aufgaben der Tätigkeit}
\section{Aufgabenzuteilung}

Im Rahmen des Projekt werden die Aufgaben von vier Mitarbeiten geteilt, meine Tätigkeiten im Team sind eine GUI Oberfläche zu entwickeln, um die Bilder und Metadaten zu steuern, XML Daten von OmniPage OCR Software und zu analysieren und Json Daten von gefundenen Objekten zu erzeugen. Einen kleinen Server zu bauen, um Die Lerndateien automatisch auszuschneiden. Einen SVM Klassifizierung - Server für Multiple Objekte zu bauen, und das rotes Typus zu erkennen. Für ganze Server Hunderttausend Lernbilder zu sortieren.

\section{Aufgabenbeschreibung}

Für weitere Details von meinen täglichen Aufgaben werden in der folgenden Abschnitten genauer beschrieben. 

\subsection{Microsoft Visual Studio}
Visual Studio ist eine für verschiedene Hochsprachen integrierte Entwicklungsumgebung, z.B. Visual Basic .Net, C, C++,C++/CLI, C++/CX, C\#, F\#, SQL Server, TypeScript und Python sowie HTML, JavaScript und CSS. Visual Studio ermöglicht es dem Programmierer, sowohl native Win32/Win64-Programme als auch Anwendungen für das .NET Framework zu entwickeln\cite{3}. Wir haben die Version 2008 und 2010 in Win32-Programme genutzt.
\subsection{OpenCV}
OpenCV (Open Source Computer Vision Library) ist eine freie Programmbibliothek, sie ist mit verschiedene Algorithmen für die Bildverarbeitung und Maschinelles Sehen zur Akademik und Kommerz unter eine BSD-Lizenz. Sie hat C++, C, Python und Java Oberflächen und unterstützt Windows, Linux, Mac OS, Android und IOS. Wir anwenden die Version 1.1, 2.1 und 2.4.10 für unsere Tätigkeiten.
\subsection{GUI: Web-Lokal-Steuerung}


\subsection{XML-Analysierung}

\subsection{Fuzzy Search}

\subsection{Rest-Service}

\subsection{Auto-Cut}
Es gibt viele Objekte auf eine Herb-beleg, z.B. Barcode, Stampe, Typus…, die wir erkennen sollen. Eine Seite kann das ganze Bild in Server (OCR) laufen, um Informationen zu extrahieren, andere Seite, es ist auch wichtig, die einzeln Objekt in Server behandeln, damit kann die Anerkennungsquote zu erhöhen. Deswegen brauchen wir riesige Menge Ausschneiden von den Objekten.
Um das Aus schneiden durchzuführen, müssen wir überlegen, wie es machen kann. Natürlich können wir das Ausschneiden manuell durch Paint oder GIMP erledigen, aber es ist offensichtlich, zeitaufwändig. Anstatt die Objekte aus Herb-Beleg selbst per Hand auszuschneiden, ist es nötig, ein kleines Programm zu entwickeln, damit kann man sehr effizient im kurzen Zeitraum viel Objekte erhalten. 
Bild ist durch das Programm hochgeladen worden, und durch Maus das Objekt abzugrenzen, z.B. bei einem Barcode klicken wir mit der linken Maustaste auf den oben links Punkt und ziehen wir bei gedrückter Maustaste bis zum unten rechts Punkt, danach lassen wir die Maustaste los. Damit bekommen wir die Koordinaten von den beiden Punkten, schneiden die Rechtangle, die die Punkte bezüglich ist, aus.\\
\begin{figure}[htbp] 
	\centering
	\includegraphics[width=0.5\textwidth]{Cutobjekte.png}
	\caption{Objekte Belege}
	\label{fig:Bild 2}
\end{figure}\\
Speichern sollen wir auch berücksichtigen, um die Objekte später direkt zum Anlernen genutzt zu werden. Brauchen wir Sortierung während Speichern, z.B. bei dem Objekt Barcode, drucken wir die Tastatur „3“, damit das Barcode würde automatisch im dritten Datei gespeichert werden. Die ausführlichen Informationen von „Objekt-Tastatur-Speicherdatei“  sind in folgende Tabelle gegeben, sehen Sie Abbildung 3.2.\\
\begin{figure}[htbp] 
	\centering
	\includegraphics[width=0.7\textwidth]{Tabelle.png}
	\caption{Objekt-Tastatur-Speicherdatei}
	\label{fig:Tabelle 1}
\end{figure}\\
Der Tabelle zeigt genau bei welchem Objekt sollen wir welche Tastatur drucken, um das Objekt richtig zu speichern.

\subsection{HSV-Farbraume}
Der HSV-Farbraum ist ein Farbraum etlicher Farbmodelle, bei denen man kann die Farbe mit Hilfe des Farbwerts (englisch hue), der Farbsättigung (saturation) und des Hellwerts (oder der Dunkelstufe) (value) definieren(Abbildung 3.3) \cite{7}. Mit diese Objekte von Typus habe ich diese Farbraum Modell wegen ihrer speziellen Farbe, normalerweise sind die immer rot, angewendet. 
\begin{figure}[htbp] 
	\centering
	\includegraphics[width=0.5\textwidth]{HSV_cone.png}
	\caption{HSV-Farbraum als Kegel}
	\label{fig:Bild 3}
\end{figure}

\subsection{Segmentierung}
Ermittlung von Rechtangle ist der Prozess, ein Bild in multiple Segmentation zu trennen. Dieser Prozess dient dafür, Analysierung von Bild und Extraktion von Merkmal zu vereinfachen.\cite{5} 
In unsere Situation ist das wichtige Merkmal von Segmentation vertikal Kante in dem Bild, der Merkmal kann ausgenutzt werden und die Regionen, die gar keine vertikale Kante haben, zu eliminieren. 
Bevor Finden der vertikale Kante, sollen wir die mögliche Rausch mit Gaussian blur möglich viele entfernen. Um die vertikale Kante zu finden, haben wir uns die Sobel Filter entschieden. Weil mit diese Funktion können wir selbst Richtungsderivat definieren. 
Nach Sobel Filter bieten wir eine Threshold Funktion, um das Bild zum Binary zu transformieren. Und danach durch Morphologische Operation können wir den Spalt zwischen Kante eliminieren und die andere nötige Kante verbinden. In unsere Situation liefert es bei der Dimension 41x30 ein gutes Ergebnis. Aber wir lassen das veränderbar, weil für uns die Bilder immer Variante ist, mit verschieden Leuchtdichte, verschiedenen Lokalisation der Objekten, manchmal sehr eng miteinander, manchmal gibt es genug Abstand zueinander, und natürlich mit verschiedenen Klarheit von den Objekten, deswegen hat die Dimension von morphologischer Operation sehr groß Beeinflusse, zu groß oder zu klein würden die Objektregionen nicht mehr finden können. Beispielweise die Abbildung 3.4).\\
\begin{figure}[htbp] 
	\centering
	\includegraphics[width=0.5\textwidth]{SVMPre.png}
	\caption{Verschneiende Bilder zu vergleichen}
	\label{fig:Bild 4}
\end{figure}\\
Links Bild mit Dimension 41x30 ist für Stampe besser, aber recht Bild ist mit Dimension 80x45 ist für Mus.Bot.Berol besser. Man kann nicht nur mit einer bestimmten Dimension alles gewonnen.
Konturen sollen dann mit dem resultierenden Bild gefunden werden. Damit erhalten wir aber viele Müll, deswegen sollen wir die Segmentation überprüfen. Mit definiertem Seitenverhältnis, In unsere Situation ist es 3.8 mit 40\% Fehlertoleranz, können wir Stampe, Barcode, Mus.Bot.Berol alle ausfinden. Die Richtung der Segmentation ist auch berüchtigt, d.h. die in beide Richtungen liegende Segmentation können herausgefunden werden.
Für Typus ist nichts anderes als Stampe, Barcode, außerdem von Beginnen werden die HSV genutzt, weil der Typus ist alles rot oder ähnlich rot. Danach mit 5x5 Dimensionen morphologische Operation, und Seitverhältnis 2.5 mit 40 Fehlertoleranz.
Die alle gefundenen möglichen Objektregionen würden auch sortiert, und parallel muss gespeichert werden. Z.B. Tastatur „0“ bezüglich auf nicht interessierte Objektregion, Tastatur „1“  bezüglich auf Stampe, Tastatur „2“ bezüglich auf Mus.Bot.Berol, und Tastatur „3“ bezüglich auf Barcode.  


\subsection{SVM-Klassifikator}
SVM ist eine Support Vector Machine, die als Klassifikator(vgl. Klassifizierung) und Regressor (vgl. Regressionsanalyse) dient. Eine SVM unterteilt eine Menge von Objekten so in Klassen, dass um die Klassengrenzen herum ein möglichst breiter Bereich frei von Objekten bleibt, sie ist ein sogenannter Large Margin Classifier (engl. „Breiter-Rand-Klassifikator“). SVMs können sowohl zur Regression als auch zur Klassifizierung verwendet werden\cite{4}.
SVMs sind keine herkömmliche Maschinen zu verstehen. Die bestehen  nicht aus greifbaren Bauteilen, sondern handeln sie sich um ein rein mathematisches Verfahren der Mustererkennung, das in Computerprogrammen umgesetzt wird. Der Namensteil Machine weist dementsprechend das Herkunftsgebiet der Support Vector Machines, das maschinelle Lernen.
In unsere Programm verwenden wir SVMs nur zur Klassifizierung, die darstellte Abbildung 3.5 zeigt, wie sollen wir die Objekte erkennen. In dem SVM Programm definieren wir drei wichtige Funktionsschritte, unser System zu trainieren.\\
\begin{figure}[htbp] 
	\centering
	\includegraphics[width=0.5\textwidth]{SVM.png}
	\caption{SVM Programm : Dieses ganze Prozesses beziehen sich auf das Programm sind Sobel Filter, Threshold Operation, Morphologic Operation, Find Contours, SVM Klassifikation, SVM Training.}
	\label{fig:Bild 5}
\end{figure}\\
Die alle gespeicherten Objektregionen, sollen wir entscheiden, dass ob die gefundene Objektregion eine Stampe oder ein Barcode ist. Um dies zu erledigen, brauchen wir jetzt SVM Algorithmus. 
Bevor der Klassifikation muss der Klassifikator mit Label trainiert werden, sog. Offline Training. In unsere Situation, wegen der vielen varianten Bilder brauchen wir eine ausreichende Menge Daten zu trainieren. D.h. je größer unsere Datenbank ist, umso kriegen wir besser Ergebnis. Deswegen brauchen wir zu mindesten ein paar tausend Herb-Beleg für Preprozessor und Segmentierung.
SVM ist inhärent ein binärer Klassifikator, in unsere Situation, für Stampe, Barcode, Mus. Bot. Berol. bedürfen wir 4 Klasse, wir nutzten lineare Klassifikation. Es gibt viele Möglichkeiten zu trainieren, z.B. one-versus-rest(OVR SVMs), one-versus-one (OVO SVMs). Wir haben uns OVR SVMs entschieden, weil für uns ist es sehr schnell und unser System ist nicht kompliziert. Unser Prozess beruht sich auf die Bildpixels, sehen Sie in folgendem Bild(Abbildung 3.6).\\
\begin{figure}[htbp] 
	\centering
	\includegraphics[width=0.7\textwidth]{Positivnegativ.png}
	\caption{Positive Bilder und negative Bilder}
	\label{fig:Bild 6}
\end{figure}\\
Klasse Stampen gegen die andere, Stampe werden mit positivem Label identifiziert, die andere wird mit negativem Label identifiziert. Barcode, Mus.Bot.Berol werden auch so trainiert. Dann die ganzen trainierten Daten  werden in eine XML Datei gespeichert.

Nach dem Bekommen der trainierten Daten müssen wir jetzt die SVM Parametern definieren, um das SVM Algorithmus zu verwenden. In OpenCv ist es einfach, dass wir können einfach CvSVMParam direkt anrufen und die Parameter eingeben. Hier könnten wir auch die Parameter automatisch mal trainieren, um die besten Parametern zu erzielen. 
Mit eingegebenen Parameters kann die Klassifikator trainieret werden. Danach ist es bereit zu Prognose von möglichen Objektregionen mit Predict Funktion. Falls wir positive Resultat erhalten, ist die Objektregion interessiert uns, negative Resultat, dann interessiert uns nicht. Somit können wir klassieren alle gefundenen möglichen Objektregionen mit der Rückmeldung von Prognose. Nur die mit dem positiven Prognose Objektregionen und auch deren Koordinaten werden gespeichert(Abbildung 3.6). \\
\begin{figure}[htbp] 
	\centering
	\includegraphics[width=0.5\textwidth]{Svmresult.png}
	\caption{Resultat von SVM Programm: Die Abbildung zeigt, wie das Resultat von SVM Programm im originalen Bild aussieht, die mit grünen Rechtangle eingehüllten Objekte sind die gefundenen Objekteregionen. Links ist für Typus, rechts ist für Stampe, Barcode, Mus.Bot.Berol.. Die gespeicherten Koordinaten werden weiter in andere Programm geliefert.
	}
	\label{fig:Bild 4}
\end{figure}

\subsection{Automatischer Objekterkennung}
Um die Koordinaten auf Herb-Beleg zu erhalten, automatische Erkennung von den Objekten ist in der erste Linien.  Dieses Programm ist genau dafür gesorgt. Es beinhaltet Ermittlung von Objekt(Preprozessor), und Klassifikation von Objekte.  Das Ziel von Ermittlung dem Objekt ist es, die möglichen Objektregionen in ganzen Bildframe zu ermitteln. Falls eine Rechtangle wird ermittelt, die Rechtangle Segment wird in den zweite Schritt übertragt, mit SVM(Support Vector Machine) Algorithmus zu entscheiden, ob die Rechtangle richtig ist. Durch das Programm können die Koordinaten von Objekt automatisch geliefert werden. Die Koordinaten können später in andere Programm eingeben, um die Region automatisch zu ausschneiden. 
\subsection{Json}

\chapter{Zusammenfassung}

\section{Erfahrungsgewinnen}

\section{Auswirkungen auf die eigene Berufsvorstellung}

\section{Bewertung der eigenen Leistungen}

\section{Rückmeldung des Arbeitgebers}

\section{Zusammenfassung der Tätigkeit}

%\chapter{Referenz}
\printbibliography
	                      	
\end{document}   